\chapter{Source Code}

This appendix contains selected source code from the FreeGA project. Complete source is available online from 
\begin{verbatim}
http://www.physics.otago.ac.nz/px/research/electronics/freega
\end{verbatim}

\section{Verilog HDL}

\begin{lstlisting}[language=Verilog]
// a module
module foo(input a, input b, output c);
endmodule

\end{lstlisting}

\section{C Code}

\subsection{Text-Mode to Raw Pixel Conversion}
The following code converts ASCII character and attribute byte information into
pixel data. This code forms the basis for the assembly versions implemented for
RISC16 and TTA16.

\begin{lstlisting}[language=C,label=C_Code_Text_Mode]
typedef unsigned int	uint32_t;
typedef unsigned short	uint16_t;
typedef unsigned char	uint8_t;

uint32_t*	fb;	// RGBA, 8-bit/component
uint16_t*	tb;	// 8-bit ASCII char + 8-bit attribute
uint8_t*	font;	// 8x16 pixel font
uint32_t*	pal;	// 16 entry colour palette

// Algorithms:
// Character at a time:
// - for each char in `tb' do:
//  - look up fg + bg colours from pal
//  - read a char from the `tb'
//  - for each row in char do:
//   - look up char row from font mem
//   - for each pixel in char row do:
//    - if pixel set, write fg colour to `fb'
//    - else write bg colour to `fb'

// FB-Row-At-A-Time:
// - for each row in FB do:
//  - calc offset into TB
//  - read char from TB
//  - look up fg + bg colours from pal
//...

// FB-Row-At-A-Time has the best SDRAM (+dcache) access behaviour?
#define	PCOLS	640
#define	PROWS	400
#define	CCOLS	80
#define	CROWS	25
#define	TCOLS	8
#define	TROWS	16


void draw_char_row(uint16_t x, uint16_t y, uint8_t ch, uint8_t at)
{
	uint16_t	i, j;
	uint8_t		r;
	i	= (((uint16_t) ch) << 4) + (y & 0x0f);
	r	= font[i];
	for (j=0; j<8; j++) {
		if (r & (uint8_t) 0x01)
			fb[y*PCOLS+x+j]	= pal[at&0x0f];	// FG
		else
			fb[y*PCOLS+x+j]	= pal[at>>4];	// BG
		r	>>= 1;
	}
}


// This assumes width of 8 pixels/char and height of 16 pixels/char and 80
// chars per row.
void draw_row(uint16_t r)
{
	uint16_t	i, j, p, idx;
	uint8_t	ch, at;

	// Calc. index into TB from row.
	idx	= (r>>4)*TCOLS;
	for (i=0; i<PCOLS; i+=8) {
		p	= tb[idx++];
		ch	= p & 0xff;
		at	= p >> 8;
		draw_char_row(i, r, ch, at);
	}
}


void update_fb()
{
	uint16_t	i;
	for (i=0; i<PROWS; i++) {
		draw_row(i);
	}
}
\end{lstlisting}


\subsection{FreeGA Kernel Module}

\begin{lstlisting}[language=C]
/***************************************************************************
 *                                                                         *
 *   freega.c - The kernel device driver module for FreeGA.                *
 *                                                                         *
 *   Copyright (C) 2006 by Patrick Suggate                                 *
 *   patrick@physics.otago.ac.nz                                           *
 *                                                                         *
 *   This program is free software; you can redistribute it and/or modify  *
 *   it under the terms of the GNU General Public License as published by  *
 *   the Free Software Foundation; either version 2 of the License, or     *
 *   (at your option) any later version.                                   *
 *                                                                         *
 *   This program is distributed in the hope that it will be useful,       *
 *   but WITHOUT ANY WARRANTY; without even the implied warranty of        *
 *   MERCHANTABILITY or FITNESS FOR A PARTICULAR PURPOSE.  See the         *
 *   GNU General Public License for more details.                          *
 *                                                                         *
 *   You should have received a copy of the GNU General Public License     *
 *   along with this program; if not, write to the                         *
 *   Free Software Foundation, Inc.,                                       *
 *   59 Temple Place - Suite 330, Boston, MA  02111-1307, USA.             *
 ***************************************************************************/

#include <linux/module.h>
#include <linux/kernel.h>
#include <linux/fs.h>
#include <linux/init.h>

#include <linux/pci.h>
#include <linux/tty.h>

#include <asm/uaccess.h>


#define	FREEGA_VENDOR_ID	0x106d
#define	FREEGA_DEVICE_ID	0x9500
#define	FREEGA_NAME		"freega"

#include "freega.h"


MODULE_AUTHOR		("Patrick Suggate <patrick@physics.otago.ac.nz>");
MODULE_DESCRIPTION	("FreeGA device driver.");
MODULE_LICENSE		("GPL");
MODULE_SUPPORTED_DEVICE	("freega");


// Globals to do with FreeGA
static int		freega_major	= 0;
static int		freega_busy	= 0;
static struct pci_dev*	freega_pdev	= NULL;
static void*		freega_mm_addr	= NULL;	// First memory mapped region
static unsigned long	freega_mm_len	= 0;
//	static

// Prototypes for file operations
static int	freega_open	(struct inode*, struct file*);
static int	freega_release	(struct inode*, struct file*);
static ssize_t	freega_read	(struct file*, char*, size_t, loff_t*);
static ssize_t	freega_write	(struct file*, const char*, size_t, loff_t*);


static struct file_operations freega_fops = {
	.read	= freega_read,
	.write	= freega_write,
	.open	= freega_open,
	.release= freega_release
};


static int freega_open (struct inode* inodp, struct file* filp)
{
	if (freega_busy)
		return -EBUSY;
	
	freega_busy++;
	try_module_get (THIS_MODULE);	// Increment usage count
//	printk (KERN_INFO "FreeGA opened.\n");
	filp->f_pos = 0;
	return 0;
}


static int freega_release (struct inode* inodp, struct file* filp)
{
	freega_busy--;
	module_put (THIS_MODULE);	// Decrement usage count
//	printk (KERN_INFO "FreeGA closed.\n");
	
	return 0;
}

/*! This implements the lseek system call. If it is left undefined, then default_llseek from fs/read_write.c is used instead. This updates the f_pos field as expected, and also may change the f_reada field and f_version field.
*/
// static loff_t (*llseek) (struct file *, loff_t, int);

/*! This is used to implement the read system call and to support other occasions for reading files such a loading executables and reading the quotas file. It is expected to update the offset value (last argument) which is usually a pointer to the f_pos field in the file structure, except for the pread and pwrite system calls.
\param off The offset for the read from the start of file
\return The number of bytes read?
*/
static ssize_t freega_read (struct file* filp, char* buf, size_t len, loff_t* off)
{
	int	notsent;
	
//	printk (KERN_ALERT "Offset = %d\n", *off);
	
/*	if (len & 0x03) {
		printk (KERN_ALERT "Reads have to be multiples of four bytes!\n");
		return -EINVAL;
	}
	
	if (*off & 0x03) {
		printk (KERN_ALERT "FreeGA read offset (%d) must be a multiple of four bytes!\n", (int)*off);
		return -EINVAL;
	}
	*/
	if ((*off + len) > freega_mm_len) {
		printk (KERN_ALERT "Cannot read past 4096 bytes!\n");
		return -EINVAL;
	}
	
	// printk (KERN_INFO "FreeGA reading.\n");
	notsent	= copy_to_user (buf, freega_mm_addr + *off, len);
	*off	+= len - notsent;
	
	return len - notsent;
}


static ssize_t freega_write (struct file* filp, const char* buf, size_t len, loff_t* off)
{
	int	notreceived;
	
/*	if (len & 0x03) {
		printk (KERN_ALERT "FreeGA writes have to be multiples of four bytes!\n");
		return -EINVAL;
	}
	
	if (*off & 0x03) {
		printk (KERN_ALERT "FreeGA write offset (%d) must be a multiple of four bytes!\n", (int)*off);
		return -EINVAL;
	}
	*/
	if ((len + *off) > freega_mm_len) {
		printk (KERN_ALERT "FreeGA cannot write past 4096 bytes!\n");
		return -EINVAL;
	}
	
	// printk (KERN_INFO "FreeGA writing.\n");
	notreceived = copy_from_user (freega_mm_addr + *off, buf, len);
	*off	+= len - notreceived;
	return len - notreceived;
}


//#ifdef __use_probe_for_freega
// Look for a FreeGA on the PCI bus, and if found, enable it.
static struct pci_dev* __init probe_for_freega (void)
{
	struct pci_dev *pdev = NULL;
	
	// Look for the FreeGA
	pdev = pci_find_device (FREEGA_VENDOR_ID, FREEGA_DEVICE_ID, NULL);
	
	if(pdev) {
		// Device found, enable it
		if(pci_enable_device (pdev)) {
			printk (KERN_ALERT "Could not enable FreeGA\n");
			return NULL;
		} else
			printk (KERN_INFO "FreeGA enabled\n");
	} else {
		printk (KERN_ALERT "FreeGA not found\n");
		return pdev;	// TODO: Is this OK?
	}
	
	return pdev;
}
//#endif


// Module initialisaton and cleanup routines.
static int __init freega_init (void)
{
	unsigned long	mm_start, mm_end, mm_flags;
	
	// More normal is printk(), but there's less that can go wrong with 
	// console_print(), so let's start simple.
	console_print("Hello, world - this is the kernel speaking\n");
	
	if ( (freega_pdev = probe_for_freega ()) )
	{
		freega_major	= register_chrdev (0, FREEGA_NAME, &freega_fops);
		
		if (freega_major < 0)
		{
			printk (KERN_ALERT "Unable to register FreeGA!\n");
			pci_disable_device (freega_pdev);
			return freega_major;
		}
		
		// TODO: Apparently UDEV can do this
		// mknod ("/dev/freega", S_IFCHR, freega_major);
		
		printk (KERN_INFO "FreeGA major number is: %d\n", freega_major);
		printk (KERN_INFO "Use: `mknod /dev/%s c %d 0'.\n", FREEGA_NAME, freega_major);
		printk (KERN_INFO "Remove /dev/%s when finished.\n\n", FREEGA_NAME);
		
		mm_start	= pci_resource_start	(freega_pdev, 0);
		mm_end		= pci_resource_end	(freega_pdev, 0);
		freega_mm_len	= pci_resource_len	(freega_pdev, 0);
		mm_flags	= pci_resource_flags	(freega_pdev, 0);
		
		if (mm_flags & IORESOURCE_MEM)
			printk (KERN_INFO "FreeGA is memory-mapped.\n");
		else {
			printk (KERN_ALERT "FreeGA memory-mapping failed!\n");
			goto cleanup0;
		}
		
		if (pci_request_regions (freega_pdev, FREEGA_NAME)) {
			printk (KERN_ALERT "FreeGA could not get requested memory mapped region!\n");
			goto cleanup0;
		}
		
		freega_mm_addr	= ioremap (mm_start, freega_mm_len);
		if (!freega_mm_addr) {
			printk (KERN_ALERT "FreeGA could not get re-map memory mapped region!\n");
			goto cleanup1;
		}
		
		printk (KERN_INFO "FreeGA mm_start = %lx\n", mm_start);
		printk (KERN_INFO "FreeGA mm_len   = %lu\n", freega_mm_len);
		printk (KERN_INFO "FreeGA mm_addr  = %lx\n", (unsigned long) freega_mm_addr);
	}
	
	// Non-zero value means failure
	return 0;
	
cleanup1:
	pci_release_regions (freega_pdev);
cleanup0:
	pci_disable_device (freega_pdev);
	unregister_chrdev (freega_major, FREEGA_NAME);
	return -1;
}


static void __exit freega_exit (void)
{
	// Un-map the allocated memory and release FreeGA
	iounmap (freega_mm_addr);
	pci_release_regions (freega_pdev);
	
	// Disable the PCI device
	unregister_chrdev (freega_major, FREEGA_NAME);
	pci_disable_device (freega_pdev);
	
	printk (KERN_INFO "FreeGA Unregistering\n");
}


module_init (freega_init);
module_exit (freega_exit);
\end{lstlisting}
\label{FreeGA_Module}


\subsection{Cache Simulator Classes}
\label{CODE_Cache}
\begin{lstlisting}
/*
 * cache.h
 *
 *  Created on: 4/12/2008
 *      Author: patrick
 */

#include "cache.h"
#include <stdio.h>
#include <stdlib.h>


// TODO: Parameterise write-thru behaviour.


WBCache::WBCache(WBMem* mem)
{
	m	= mem;
	// Two banks of tags and memory.
	t0	= new Tag[CACHE_TAGNUM];
	b0	= new uint16_t[CACHE_BANKSIZE];

	t1	= new Tag[CACHE_TAGNUM];
	b1	= new uint16_t[CACHE_BANKSIZE];

	lru	= new bool[CACHE_TAGNUM];
	for (int i=0; i<CACHE_TAGNUM; i++)
		lru[i]	= false;

	p_tag0	= 0;
	p_idx0	= 0;
	p_vld0	= false;

	p_tag1	= 0;
	p_idx1	= 0;
	p_vld1	= false;

	accesses= 0;
	misses	= 0;
	hits	= 0;
	slow	= 0;
	evicts	= 0;
}


WBCache::~WBCache(void)
{
	delete[] t0;
	delete[] b0;
	delete[] t1;
	delete[] b1;
}


// Costing:
// - A fast-hit only costs two CPU cycles, a slow-hit takes three.
// - Due to refreshing and congestion, a single word fetch costs about 40
//   cycles on average. A whole cache-line is fetched though, at 1/3 of the
//   CPU clock rate, but at 32-bits/per transfer. An evict requires the
//   contents of the cache to be written back first. If this is buffered,
//   then there is probably no penalty in most situations.
void WBCache::stats(void)
{
	float	rate	= (float)misses / (float)accesses * 1000.0f;
	float	fast	= (float)(hits-slow) / (float)(hits) * 100.f;
	printf("Cache Statistics:\n");
	printf("Accesses:\t%lu\n", accesses+wrtc);
#ifdef __use_write_thru
	printf("Write Thrus:\t%lu\n", wrtc);
	printf("Read Hits:\t%lu (%2.1f%%)\n", hits, (float)hits/(float)accesses*100.0f);
#else
	printf("Hits:\t\t%lu (%2.1f%%)\n", hits, (float)hits/(float)accesses*100.0f);
#endif
	printf("Fast Hits:\t%lu (%2.1f%%)\n", hits-slow, fast);
	printf("Misses:\t\t%lu\t(evicts = %lu)\n", misses, evicts);
	printf("Miss rate(per 1000):\t%3.1f\n\n", rate);
#ifdef __use_write_buffer
	uint64_t	wrthru_cost	= FAST_COST;
#else
	uint64_t	wrthru_cost	= AV_LATENCY*CPU_MULT;
#endif
	uint64_t cost	= (hits-slow)*FAST_COST +
			  slow*SLOW_COST +
			  wrthru_cost*wrtc +
			  (misses + __evict_cost*evicts)*(SLOW_COST+AV_LATENCY-1+CACHE_LINESIZE/4*CPU_MULT);
	printf("Total CPU memory access cycles:\t%lu (Av: %1.1f/op)\n\n\n", cost, (float)cost / (float)(accesses+wrtc));
}


// 1/ Check cache
// 2/ If hit:	return data
//    Else:	Fetch cacheline
//		return data
//
uint16_t WBCache::read(int a)
{
	uint32_t	tag	= CACHE_TDATA(a);
	uint32_t	idx	= CACHE_TIDX(a);
	uint32_t	off	= CACHE_TOFF(a);
	
	hits++;
	accesses++;

#ifdef __use_rand_evict
	bool	new_lru	= (rand()&0x1) ? true : false ;
#else
	bool	new_lru	= !lru[idx];
#endif

#ifdef __use_fast_path
	if (p_tag0 == tag && p_idx0 == idx && p_vld0) {
#ifndef __use_fifo_evict
		lru[idx]	= true;
#endif
		return	b0[CACHE_ADDR(idx, off)];
	}
	if (p_tag1 == tag && p_idx1 == idx && p_vld1) {
#ifndef __use_fifo_evict
		lru[idx]	= false;
#endif
		return	b1[CACHE_ADDR(idx, off)];
	}
#endif	// __use_fast_path

	slow++;

	// Slower hit path.
	if (t0[idx].hit(tag)) {
		p_tag0	= tag;
		p_idx0	= idx;
		p_vld0	= true;
#ifndef __use_fifo_evict
		lru[idx]	= true;
#endif
		return	b0[CACHE_ADDR(idx, off)];
	}

	if (t1[idx].hit(tag)) {
		p_tag1	= tag;
		p_idx1	= idx;
		p_vld1	= true;
#ifndef __use_fifo_evict
		lru[idx]	= false;
#endif
		return	b1[CACHE_ADDR(idx, off)];
	}

	hits--;
	slow--;
	misses++;

	// Miss and fetch path.
	// Is the old data dirty? If so evict.
	if (lru[idx] && t1[idx].wrback())
		evict(b1, &t1[idx], idx);
	else if (!lru[idx] && t0[idx].wrback())
		evict(b0, &t0[idx], idx);

	uint16_t	rv;
	if (lru[idx])	{
		rv	= fetch(b1, &t1[idx], a);
		p_tag1	= tag;
		p_idx1	= idx;
		p_vld1	= true;
		lru[idx]= new_lru;
	} else {
		rv	= fetch(b0, &t0[idx], a);
		p_tag0	= tag;
		p_idx0	= idx;
		p_vld0	= true;
		lru[idx]= new_lru;
	}

	return	rv;
}


// Write through is simpler and avoids some coherency issues, but I suspect
// that it will be a lot slower.
void WBCache::write(int a, uint16_t dat)
{
	uint32_t	tag	= CACHE_TDATA(a);
	uint32_t	idx	= CACHE_TIDX(a);
	uint32_t	off	= CACHE_TOFF(a);

#ifdef __use_write_thru
	wrtc++;
	m->putw(a<<1, dat);
#else
	hits++;
	accesses++;
#endif
	
#ifdef __use_rand_evict
	bool	new_lru	= (rand()&0x1) ? true : false ;
#else
	bool	new_lru	= !lru[idx];
#endif

	// Fast hit path.
#ifdef __use_fast_path
	if (this->p_tag0 == tag && this->p_idx0 == idx && this->p_vld0) {
		this->b0[CACHE_ADDR(idx, off)]	= dat;
#ifndef __use_fifo_evict
		lru[idx]	= true;
#endif
#ifndef __use_write_thru
		this->t0[idx].mark();
#endif
		return;
	}

	if (this->p_tag1 == tag && this->p_idx1 == idx && this->p_vld1) {
		this->b1[CACHE_ADDR(idx, off)]	= dat;
#ifndef __use_fifo_evict
		lru[idx]	= false;
#endif
#ifndef __use_write_thru
		this->t1[idx].mark();
#endif
		return;
	}
#endif	// !__use_fast_path
	
#ifndef __use_write_thru
	slow++;
#endif
	// Slower hit path.
	if (this->t0[idx].hit(tag)) {
		this->p_tag0	= tag;
		this->p_idx0	= idx;
		this->p_vld0	= true;
#ifndef __use_fifo_evict
		lru[idx]	= true;
#endif
		this->b0[CACHE_ADDR(idx, off)]	= dat;
#ifndef __use_write_thru
		this->t0[idx].mark();
#endif
		return;
	}

	if (this->t1[idx].hit(tag)) {
		this->p_tag1	= tag;
		this->p_idx1	= idx;
		this->p_vld1	= true;
#ifndef __use_fifo_evict
		lru[idx]	= false;
#endif
		this->b1[CACHE_ADDR(idx, off)]	= dat;
#ifndef __use_write_thru
		this->t1[idx].mark();
#endif
		return;
	}

#ifndef __use_write_thru	// Stalls processor if no write buffer, so count as miss.
	hits--;
	slow--;
	misses++;	// Write-backs can miss too
	
	// Miss and fetch path.
	// Is the old data dirty? If so evict.
	if (lru[idx] && t1[idx].wrback())
		this->evict(this->b1, &this->t1[idx], idx);
	else if (!lru[idx] && t0[idx].wrback())
		this->evict(this->b0, &this->t0[idx], idx);

	if (lru[idx])	{
		fetch(b1, &t1[idx], a);
		this->t1[idx].mark();
		b1[CACHE_ADDR(idx, off)]	= dat;
		p_tag1	= tag;
		p_idx1	= idx;
		p_vld1	= true;
		lru[idx]= new_lru;
	} else {
		fetch(b0, &t0[idx], a);
		this->t0[idx].mark();
		b0[CACHE_ADDR(idx, off)]	= dat;
		p_tag0	= tag;
		p_idx0	= idx;
		p_vld0	= true;
		lru[idx]= new_lru;
	}
#endif
}	// write


void WBCache::evict(uint16_t* b, Tag* t, int idx)
{
	int	a	= CACHE_ADDR(idx, 0);
	uint32_t	d;
	evicts++;

	for (int i=0; i<CACHE_LINESIZE/4; i++) {
		d	= (uint32_t)b[a];
		d	|= ((uint32_t)b[a+1])<<16;
		uint32_t	adr	= CACHE_MEMADDR(t->tag(), a);
		this->m->putl(adr, d);
		a	+= 2;
	}
	t->invld();
}


uint16_t WBCache::fetch(uint16_t* b, Tag* t, int adr)
{
	uint32_t	d, a;
	int	p	= CACHE_ADDR(CACHE_TIDX(adr), 0);
	a	= (adr & ~CACHE_OFFMASK) << 1;

	for (int i=0; i<CACHE_LINESIZE/4; i++) {
		d	= this->m->getl(a);	// Convert word address to byte address
		b[p++]	= (uint16_t)d;
		b[p++]	= (uint16_t)(d>>16);
		a	+= 4;
	}
	t->set(CACHE_TDATA(adr));
	return	b[CACHE_ENTRY(adr)];
}


// Assume DWord alignment of `p'.
uint32_t WBCachedMem::getl(int p)
{
	uint32_t	d;
	d	= c->read(p>>1);
	d	|= ((uint32_t)c->read((p>>1)+1))<<16;

	return	d;
}


uint8_t WBCachedMem::getb(int p)
{
	uint16_t	d;
	d	= c->read(p>>1);
	if (p&0x1)	return	(uint8_t)(d >> 8);
	else		return	(uint8_t)d;
}


// Assume DWord alignment of `p'.
uint32_t WBCachedMem::putl(int p, uint32_t v)
{
	c->write(p>>1, (uint16_t)v);
	c->write((p>>1)+1, (uint16_t)(v>>16));
	return	v;
}


// No byte access, so read before write.
uint8_t WBCachedMem::putb(int p, uint8_t v)
{
	uint16_t	t	= c->read(p>>1);
	if (p&0x1)	c->write(p>>1, (t&0x00ff) | ((uint16_t)v<<8));
	else		c->write(p>>1, (t&0xff00) | (uint16_t)v);
	return	v;
}


void WBCachedMem::push(int n)
{
	for (; n>0; n--)
		this->c->write(sp-->>1, rand());
}


void WBCachedMem::pop(int n)
{
	for (; n>0; n--)
		this->c->read(sp++>>1);
}


void dump_defines(void)
{
	printf("OFFMASK\t = %x\n", CACHE_OFFMASK);
	printf("IDXMASK\t = %x\n", CACHE_IDXMASK);
	printf("DATAMASK\t = %x\n", CACHE_DATAMASK);

	printf("LINESIZE\t = %d\n", CACHE_LINESIZE);
	printf("TAGNUM\t = %d\n", CACHE_TAGNUM);
}


void WBCachedMem::stats(void)
{
	uint64_t	r, w;
	
	printf("\nCache Properties:\n");
	printf("\tCache Size:\t\t%6d\n", CACHE_MEMSIZE);
	printf("\tTags per Bank:\t\t%6d\n", CACHE_TAGNUM);
	printf("\tSet associativity:\t 2-way\n");
	printf("\tLine (Block) Size:\t%6d\n", CACHE_LINESIZE);
#ifdef	__use_write_thru
	printf("\tWrite Policy:\t\t  thru\n\n");
#else
	printf("\tWrite Policy:\t\t  back\n\n");
#endif
	
	printf("Memory Statistics:\n");
	printf("Reads:\t\t%lu\n", r=m->num_reads());
	printf("Writes:\t\t%lu\n", w=m->num_writes());
	printf("Total:\t\t\t%lu\n\n", r+w);
	
	c->stats();
}
\end{lstlisting}

\section{TTA Assembly}



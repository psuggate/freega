\chapter{Introduction}
This thesis presents the design and implementation of a
\gls{fpga}-based,
free and open-source hardware, graphics adapter. Free and Open-Source Software
(\gls{foss}) has
achieved widespread use and acceptance~\cite{lerner2002sse} though open-source
designs of the underlying hardware which comprise the systems on which the
software is executed have not. While the interfaces between many components have
been standardised, the implementation of these components is often proprietary,
and considered a trade secret.

Graphics adapters are an example of proprietary hardware. They have to be
compatible with the \gls{vga} (Video Graphics Array), or an earlier \gls{ibm} specification, to function as a
\gls{pcomp}'s primary adapter. This
specification has never been released, all current implementations are
proprietary. This leads to significant barriers for those wishing to create their
own, whether for research or commercial purposes.

By implementing a design within programmable logic, such as an FPGA, it is
possible for organisations, and even individuals, to design and implement complex
digital circuits without having to pay the significant Non-Recurring
Engineering (\gls{nre}) costs
to have an Application Specific Integrated Circuit (\gls{asic}) fabricated. By
using logic cores, advanced digital circuits can be constructed within
Programmable Logic Devices (\gls{pld}s) by combining many smaller cores. With the advent of high logic
capacity, low-cost FPGA families\footnote{High-capacity, low-cost FPGA product
ranges that are readily available at the time of writing are Xilinx Spartan,
Altera Cyclone, and Lattice EC, and other vendors have similar products too.},
and with freely available \gls{hdl} EDA tools, the barriers to designing and
implementing advanced logic cores are now significantly lower. And like FOSS, it
is possible for the Register Transfer Level (\gls{rtl}, the most popular method for
describing digital logic circuits in HDL) descriptions of logic cores to be made
freely available, resulting in free and open-source hardware designs.


\section{Purpose of this Project}
The purpose of this project, \gls{openvga}, is to
use low-cost FPGAs to implement an open-source, FPGA-based, computer graphics
adapter. The HDL, driver and firmware source-code, and \gls{pcb} design are to be freely available. The
objective is to lower the obstacles that exist for developing PC display
adapters. Additional goals for OpenVGA are to:

\begin{itemize}
  \item Develop HDL logic cores and implement a very flexible display
  adapter, along with a collection of tools and testbenches. The design should
  allow a useful subset of the VGA specification to be emulated in the future.
  \item Produce Operating System (\gls{os}) device drivers so that OpenVGA can be recognised by the OS, and
  used by software applications.
  \item Use commonly available components to allow a moderately low-cost adapter
  to be produced.
  \item Have all source code and PCB artwork generated by this project released
  under the GNU General Public License (\gls{gpl}), and available on the
  Internet. This will allow others to modify and extend OpenVGA, and
  contribute back the changes as well.
\end{itemize}


\section{Relevance of OpenVGA}
Since OpenVGA is an open graphics development platform, and with a small
logic-use footprint, it could be ideal as the basis for other graphics projects
(like a real-time ray tracer~\cite{TTA_Ray_Trace} or for data visualisation). It
could possibly even serve as a low-cost development board for a hardware
computation project, since it contains a FPGA, a Peripheral Component
Interconnect (\gls{pci}) Local Bus connector, and some \gls{dram}.

At the time of writing, the availability of high-quality, open-source logic cores
is limited\footnote{OpenCores seems to be the largest repository of open-source
logic cores, and it hosts many interesting projects, but these are of variable
quality and many more cores are needed.}, especially when compared to the number
of FOSS projects. OpenVGA adds to this pool of logic cores as it contains logic
cores which can be of use to other projects. Possibly useful logic cores are: a
couple of small and fast processors, \gls{tta16} and \gls{risc16}; a simple data cache; a Synchronous
Dynamic Random Access Memory (\gls{sdram}) controller; a simple PCI to Wishbone bridge; and VGA and
\gls{dvi} redraw logic
cores.


\section{Limitations of this Project}
% TODO: Proofread, too many footnotes?
OpenVGA was not intended to have advanced features and compete with graphics
adapters developed by the major vendors; currently Intel, AMD, and Nvidia. These
companies employ hundreds of engineers to design and test their graphics
adapters. These modern graphics adapters are extremely sophisticated, containing
many millions of logic gates, and have 2D and 3D hardware acceleration
functionality\footnote{A commercial graphics adapter like an AMD
Radeon\texttrademark HD 4890 claims to have floating-point calculation
performance of over one teraFLOP, and can render millions of triangles per
second, accelerate the play-back of encoded video streams, and has a memory
data-bus that is 256-bits wide~\cite{AMD_4890}.}. Programmable logic does not
allow designs as large, or that can achieve the same operating frequencies, as
those implemented as ASICs. OpenVGA is designed to be a simple, open-source,
graphics adapter instead.

At the time of writing, OpenVGA cannot function as a PC's primary graphics
adapter, since it would need to be compatible with the VGA specification. Meeting
the VGA specification was beyond the scope of this project, though OpenVGA was
designed to support VGA emulation, and this is a future goal of the open-source
OpenVGA project. Already though, once an OS is loaded, OpenVGA can be accessed
using a software device driver.

Another limitation of OpenVGA is that there are only two frequencies available as
the dot-clock for the display . This is the clock used to generate display timing
signals and draw \gls{pixel}s\footnote{Pixel is a simplification of the two words
``picture element''.} to the screen. This limits the range of supported
screen resolutions. Due to the Spartan-3 architecture, an external clock
generator IC would be needed to allow for a large range of supported display
resolutions.

To meet the design objectives of low-cost and easy to fabricate meant using a
two-layer PCB. This limited data-bus widths and frequencies between the Spartan-3
FPGA and its associated peripherals. The memory bus-width was constrained to
16-bits because of the limited number of Input/Output (\gls{io}) connections
available on non-Ball Grid Array (\gls{bga}) Spartan-3 FPGAs.
Xilinx recommends a PCB with six or more layers for its BGA-packaged FPGAs.

The two-layer PCB also reduced performance within the high-frequency sections of
the design as well. For example, the maximum stable SDRAM operating frequency
achieved by the memory hardware testbench was 120 MHz, though the manufacturer
specifies 133 MHz for the SDRAM IC used. This was due to limited options for
placing ground and power planes, and also placing termination resistors is more
difficult. The result was that signal integrity issues limited the speed of some
signals to below those listed in the manufacturer's specifications.

Due to this combination of lower operating frequencies and narrower bus widths,
compared with current commercial graphics adapters, OpenVGA has a relatively low
memory bandwidth (the peak measured during testing was 240 MB/s). This limits
available display modes and many possibilities for hardware
acceleration\footnote{There is an FPGA processor logic core implemented as part
of the OpenVGA design. This is for data processing tasks and managing the state
of the graphics adapters (setting video modes and initialisation). This processor
is pipelined and operates at 150 MHz, features instruction-level parallelism, and
has a high-speed data cache. A future version of OpenVGA firmware will use this
processor for some 2D acceleration tasks as well as emulating VGA.}.


\section{Outline of this Thesis}
A review of PC graphics adapters, both past specifications and current related
projects, is presented in Chapter~\ref{BACKGROUND}. This introduces graphics
adapters developed for the computer architectures derived from the original IBM
PC. Included is a brief overview of the history and development of PC graphics
adapters up until VGA. These adapters share many functional components, with each
succeeding adapter generation building on the specifications of previous one.
These functional components, and their operation, are detailed. This chapter then
concludes with a review of other open-source graphics adapter projects.

% Chapter 3
The architecture of the graphics adapter presented within this thesis, OpenVGA,
is introduced in Chapter~\ref{OPENVGA}. Topics covered are the OpenVGA hardware
and an introduction to the logic cores that were developed. These logic cores are
used to provide the functionality within the FPGA, which is the core component of
this graphics adapter.

% Chapter 4
OpenVGA can use one of two custom, FPGA-based processors for initialisation, mode
management, and data processing tasks. Chapter~\ref{CPU} first presents an
overview of processor architectures and other processor design topics. Included
is a discussion of a novel technique for incrementing a processor's Program
Counter (\gls{pc}), and an uncommon, but powerful, class of processor
architecture, the Transport Triggered Architecture (\gls{tta}). The body of this
chapter then covers the development of two processors
with radically different architectures, TTA16 and RISC16. Concluding the chapter
is a comparison of these two processors and how well suited they are for OpenVGA.

% Chapter 5
Graphics adapters have local memory and Chapter~\ref{MEM} presents the logic
cores for accessing and caching this local memory. The local memory is used for
storing display data, firmware, and adapter state information. OpenVGA has both
ROM and RAM and a cache is used to give the OpenVGA processor fast, low-latency
access to this memory. The design of the DMA controller used within OpenVGA is
discussed as well. It was developed so that the processor can then write data
back to the RAM using efficient burst transfers.

% Chapter 6
Interfacing the significant functional components of OpenVGA, both within the
FPGA and the FPGA to external interfaces, the numerous modules needed for this,
and drawing to the display, are the topics of Chapter~\ref{IO_Chapter}. Since
different components operate at different clock frequencies, data synchronisation
problems had to be solved and the solutions are presented in Section~\ref{CLOCK}.

% Chapter 7
The project summary and conclusions are discussed in Chapter~\ref{CONCLUSION}.
This is a summary of the current status of OpenVGA and what has been achieved, it
includes comparisons of the two processors developed and how they compared to
existing processors. Finally, areas for future work are covered, including
topics such as software drivers, firmware, logic core improvements, DVI support
and testing, and upgraded hardware.

% Appendices
This thesis includes several appendices that cover important information that is
related to this project. There is a source code overview in
Appendix~\ref{Source_Code}. The code described here is available from an Internet
open-source software repository SourceForge (http://openvga.sourceforge.net/).
The hardware components and PCB artwork is included in Appendix~\ref{HARDWARE}.
The Wishbone interconnect is described in Appendix~\ref{APP_Wishbone}, and
Appendix~\ref{TTA_Programming} and Appendix~\ref{RISCPROG}, are the assembly
programming guides for the TTA16 and RISC16 processors.

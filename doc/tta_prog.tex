\chapter{TTA16 Assembly Language Programming Guide}
\label{TTA_Programming}
% OpenVGA is designed to emulate VGA functionality using a reasonably
% general-purpose processor, either TTA16 or RISC16 .

Programming for the TTA16 architecture, and accessing the available OpenVGA
hardware using the processor, requires significant documentation. This is
because many modules have been created specifically for OpenVGA, so no
documentation exists elsewhere, and the TTA16 architecture is unusual. This
appendix does not attempt to provide exhaustive information, but hopefully
enough to understand the source-code in the examples, and the firmware routines
if necessary.


\section{TTA16 Overview}
Programming for the TTA16 architecture is unlike programming for more
traditional architectures since a TTA with multiple transports has explicit-ILP
(Instruction Level Parallelism). TTA16 instruction consists of multiple
micro-instructions, and each being simple data moves between registers. A
traditional CPU instruction typically specifies just a single operation to be
performed -- the \textit{opcode} -- and the arguments: registers, immediate
constants and memory data.

A TTA processor contains three directly accessible register types: trigger
registers, operand registers, and results registers. A write to a trigger
register initiates an operation within its associated FU. A write to an operand
register has no side-effect, that value is simply changed. Operand registers
are used when multiple inputs to a FU are required. Subtract, for example, has
two inputs, the minuend and the subtrahend. With TTA16, the minuend is the
trigger register and the subtrahend input is an operand register.

The output of the subtract unit is a result register. These registers are
modified by its associated FU a certain number of cycles, depending on the FU,
after the trigger register is written to. These results can then put back onto
a data transport to be used for subsequent operations.

Here is an example subtract operation for a TTA processor with one transport:
\begin{center}
\texttt{
\begin{tabular}{l l l}
\{r0	& ->add		& \}	\\
\{r1	& ->add$_\mathtt{t}$	& \}	\\
\{diff	& ->r2		& \}
\end{tabular}
}
\end{center}
As part of the TTA assembler's syntax, curly braces (\{\}) surround an
instruction. The reason for syntax is to make obvious the explicit-ILP of the
processor when the processor has multiple transports (see the \texttt{memcpy}
programming example in Figure~\ref{TTAPROG_TTA16_memcpy} at the end of this
appendix for assembly code written for TTA16 which has multiple transports).

On the first line, the argument \texttt{r0} is a general purpose register
from the RF (Register File), one of 16 (\texttt{r0-r15}). The arrow \texttt{->}
indicates the data direction, from \texttt{r0} to \texttt{add} in this case.
The last argument is the destination register, which is the operand register
for the addition FU. Since \texttt{add} is an operand register, this is a
simple data move, there is no side-effect, so the start of an addition will not
be triggered.

The second line moves another register to the addition FU, but this time to the
trigger register. This instruction will cause an addition operation to begin.
The sum of the two input registers will be stored back to the RF by the third
instruction.

The programming tasks with TTA16, especially with high-performance
code, are more explicitly concerned with scheduling the flow of data to and from
FUs (Functional Units) via `transports'. TTA16 is not fully connected either,
each transport can only read certain registers, and direct data to just a subset
of total FUs, which increases programming complexity. 


\subsection{TTA16 Datatypes}
The only native data types supported by TTA16 are 16-bit, signed and unsigned
words. There is a little hardware support for larger word sizes though. This
is achieved using the borrow flag for long subtractions, and the multiplier
calculates a 32-bit product from two 16-bit inputs. Either half of the product
can be read, just as any other 16-bit register.

Immediate constants are either 8-bits, sign-extended to 16-bits, or 11-bits
sign-extended (which is only available to transport 0 as it is needed for
branching). Producing longer immediate values requires multiple operations, for
example using the multiplier as a shifter.


\subsection{Instruction Format}

Like RISC processors, TTAs tend to have fixed width instruction words
\cite{corporaal1993maa}. Typically within an instruction word, all fields are
predefined, whereas RISC tends to have multiple instruction
formats\cite{SPARC_Arch,smith1994paa,britton2004mal,ARM_Cortex_M3}. A
disadvantage with TTAs is that instruction words are often wider, or loading
large immediate data values is more complex\footnote{For example, MOVE32INT, a
TTA CPU designed at TU Delft by H. Corporaal et. al. has only 6-bit immediates,
but upto four can be used per instruction\cite{corporaal1993maa}.}.

\begin{figure}[h!]
\begin{center}
\includegraphics[width=\linewidth]{diagrams/tta16_instr.eps}
\caption[TTA16 instruction format]{TTA16 instruction format, see text for a
complete explanation of the bit-fields.}
\label{TTAPROG_Instruction_Format}
\end{center}
\end{figure}

TTA instructions consist of a list of data moves to complete during the
transport stage of the pipeline, and often some extra information like
guards\cite{corporaal1993maa}, register file indices, and/or immediate data.

The TTA16 instruction word format is shown in
Figure~\ref{TTAPROG_Instruction_Format}. The fields labelled \textit{SRCx},
where \textit{x} is 0, 1, or 2, are the bit-fields which select the
source register for the data-transport. The \textit{DSTx} bit-fields specify
the destination registers. The \textit{COM} register is a special source
register that is visible to all other transports, and is an operand register
for most FUs.

The next fields select registers from the RF to read and/or write. The single
\textit{B} bit selects the bank, i.e. registers 0-7 or 8-15 for the
instruction. Since register write index bit-fields follow the current
instruction, and read bit-fields precede it, a register read and write
instruction involving different banks is still possible, like the following
part-instruction (a valid move for transport 3):
\begin{center}
\texttt{r0	-> r15}
\end{center}

The final bit-field is the immediate data field. This is 8-bits wide, with the
upper bit representing the sign for sign-extending the integer to 16-bits. Data
transport 0 uses an 11-bit immediate, by using the value of \textit{REG1} as
well, so that branches can reach all locations within the TTA16 instruction
block RAMs.

Only a subset of TTA16's registers are available to each data transport (see
Table~\ref{TTAPROG_Registers}). This is so that the internal multiplexers can be
smaller, and fewer logic layers, resulting in a more efficient implementation.
Since the \textit{com} register can read most registers, and all transports can
read \textit{com}, it is believed that this restricted set of registers,
accessible to each transport, is a good trade-off between size and clock-speed,
and code performance.

An example move for transport 0 could be:
\begin{center}
\texttt{0x023	-> bra}
\end{center}
Which means move the hexadecimal value 0x023 (in base-10, 35) to the
unconditional branch register, i.e. jump to address 0x023 and start executing
instructions there. Since there is a latency of two cycles, the instruction
follwing this branch instruction would be executed before the branch is
completed too.

From transport 0, it would not be possible to write 0x023 to the \textit{sub}
register, for example, since this register is not visible to this transport. But
it would be a valid move when performed in transport 1 .

\begin{table}[h!]
\begin{center}
\begin{tabular}{l l | l l | l l | l}
SRC0	& DST0	&	SRC1	& DST1	&	SRC2	&	DST2	&	COM	\\
\hline
com		&	nop	&	com		&	nop	&	com		&	mem/nop	&	com/nop	\\
IMM11	&	bra	&	IMM8	&	sbb	&	diff	&	RF0		&	IMM8\\
RF1		&	rad	&	RF1		&	sub	&	RF0		&	mul		&	RF0	\\
mem		&	wad	&	mem		&	cmp	&	pc		&	msr		&	mem	\\
		&	jb	&	nand	&		&			&			&	bits\\
		&	jnb	&	and		&		&			&			&	diff\\
		&	jz	&	or		&		&			&			&	plo	\\
		&	jnz	&	xor		&		&			&			&	phi	\\
\end{tabular}
\end{center}
\caption[TTA16 registers and the transports to which they have access]{TTA16
registers and the transports which they are accessible.}
\label{TTAPROG_Registers}
\end{table}


\subsection{Data Transport Scheduling}
All FUs, except the Wishbone interface FU, have a fixed latency, so a result is
available a deterministic number of cycles following triggering, and will be
over-written by subsequent FU triggering, after its fixed latency has elapsed.
For example, the subtract FU has a latency of two cycles, as do all FUs except
the register file (RF), this has a latency of three due to pipelining. Even the
branch FU has a latency of two, causing the instruction following the branch to
be executed. The instruction slot directly after a branch instruction is called
a branch delay slot\cite{SPARC_Arch}.

If a FU is read one cycle after it was triggered, triggering is by a write to a
FUs trigger register (TR), it will still contain the previous value. Traditional
architectures often stall the CPU until the result is available, or use data
forwarding to get early access to the result. This approach was not taken with
TTA16 for two reasons:
\begin{enumerate}
  \item The extra hardware required for hazard detection, CPU stalling,
  and/or data forwarding would lead to a larger CPU, reducing one the key
  benefits of TTA processors.
  \item It is not always desirable, especially with RF bandwidth being
  comparatively more scarce than other architectures. Taking advantage of this
  latency allows `time-sharing' of a FU, like a subtractor, for the decrementing
  of two variables without having to write either result back to the register
  file.
\end{enumerate}

There is one exception though, TTA16 does stall when a Wishbone access is
initiated, resuming upon completion of the transaction. Though the interlock
logic required to implement this carries a performance penalty, it is less than
a more general data hazard detection scheme, and Wishbone accesses are
non-deterministic. A low silicon-cost solution would be to poll a machine
status register until a Wishbone transaction completion signal is received, but
this would lead to a large increase in code size.

Traditional architectures use the RF extensively, as does RISC16 (see
Appendix~\ref{RISCPROG}), typically two register reads and one write per
instruction. While TTA16 has a RF, with a capacity of 16, 16-bit values, typical
code only accesses the RF only about once per instruction (see
Listing~\ref{TTAPROG_TTA16_memcpy} for some typical assembly code). This is
because moving data between the RF and FUs uses up available processor data-move
bandwidth. It is more efficient to move data from one FU straight to another.

While not often used, TTA16 is still capable of up to two register reads, one
write, and using one unique immediate data value within a single instruction,
similar to more traditional architectures. There are additional restrictions
though, both register reads have to be from the same bank, register field
sharing, and register fields from the preceding and following instructions
determine the registers read and written respectively (see
Figure~\ref{TTAPROG_Used_Fields}).

Examining the instruction word diagram in
Figure~\ref{TTAPROG_Instruction_Format}, it may be noted that there are only two
register fields within the instruction word. The first two transports (0 and 1)
share \textit{REG1} and the last two transports (2 and \textit{COM}) share
\textit{REG0}, with \textit{REG0} optionally specifying the write register.

\begin{figure}[h!]
\begin{center}
\includegraphics[width=\linewidth]{diagrams/tta16_used_fields.eps}
\caption[TTA16 instruction shared bit-fields]{TTA16 shared bit-fields between
instructions. A TTA16 instruction uses the \textit{B, REG0} and \textit{REG1}
fields from the previous instruction for register reads, the \textit{REG0}
field from the next instruction for register writes, and there is overlap of
the \textit{IMMED} and \textit{REG1} bit-fields when using immediates with
transport 0. Long immediates are so branches can reach every address in block RAM.}
\label{TTAPROG_Used_Fields}
\end{center}
\end{figure}

Additionally, it is the register fields of the preceding instruction word
that contain the indices of registers to be read for the current instruction,
and the following instruction contains the index of the register to be written.
Because of these architectural peculiarities, register operations can be
difficult to schedule correctly. Often it is not until code assembly is
attempted that conflicts are discovered.

% TODO: This doesn't belong in a manual.
Due to the explicit ILP, interleaved instruction bit-fields, and only partial
data-transport connectivity, tightly-packed, well-scheduled code is very time
consuming to construct by hand. No compiler or other tools for instruction
scheduling were written or readily available to ease this task either. The
main goal when implementing the OpenVGA firmware was to ensure that critical
loops are highly optimised (like with the memory copy subroutine, see
Listing~\ref{TTAPROG_TTA16_memcpy}).


\subsection{Exceptions}
TTA16 has no support for exceptions due to the complexity of saving and
restoring the state of a TTA processor. For example, some registers of the TTA16
processor are write only which means that these functional units would be
difficult to use during an interrupt. The multiply FU, for example, uses a
dedicated, write-only, triggering register as one input, the TTA16 COM register
for the other, so it could be possible to work backwards to deduce the value
within the triggering at the time of the interrupt, but the processor would
need its own set of working registers while performing the calculations.

Exception complexity was avoided by avoiding exceptions/interrupts altogether.
The TTA16 within OpenVGA does not need to respond immediately to any I/O
requests, PCI requests go straight to local memory, so exceptions are not needed


\section{A Simple Example}
The following is a piece of TTA16 assembly code which sets OpenVGA's LEDs.
Braces surround a complete instruction, commas separate the different execution
transports, which control the data transports, and there are four streams per
instruction. Each stream has access to just a subset of TTA16's total register set.

\footnotesize
\verbatiminput{source/tta16_set_leds.tex}
\normalsize

% TODO: Step-by-step explanation.
The first two lines are just comments, since the line begins with the `;'
character, and any text after this -- until the end of the line -- is ignored.
The assembler supports two types of comments, single-line and multi-line
comments, which is any text between the comment opening sequence `/*' and the
closing sequence `*/'.

This is a line-by-line break-down of the above example, only lines with
instructions (\{\ldots\}) are counted:

\begin{enumerate}
  \item This line begins with a label so that branches to this routine can use
  a convenient moniker, rather than using the exact numerical value of the
  address. It would be difficult to determine this address while writing code 
  too, as its value is only generated during assembly (using a pseudo-random
  count order calculated by the MFSR used).  \\
  The purpose of this line is to load the number one into the \texttt{com}
  register, to be used by the following line. Note that fields can be left blank and the
  assembler automatically inserts no-operations (the \texttt{nop} opcode can
  be explicitly used too) for that transport. The commas are required though.
  \item The numerical value of one is now within \texttt{com}, placed there
  by the previous instruction, and by moving the contents of \texttt{com} to the
  \texttt{msr} register, the desired MSR (Machine Special Register) is
  selected, in this case it is the write address segment register. This
  register is used whenever a Wishbone write transaction is initiated, the
  lower seven bits of this register form the upper seven bits of the 23-bit
  Wishbone address. \\
  The final field in the instruction is the segment address for the LEDs FU,
  and this is the value that will be loaded into MSR \#1, the write address
  segment register.
  \item There are three used fields within this instruction, the left-most
  operation is a data move to the write address register (\texttt{wad}), which
  is a trigger register, to initiate a Wishbone write transaction. This value will
  be ignored by the LEDs FU, as it does not use the lower 16-bits of the
  Wishbone address field, so any value can be used. It isthe segment address value that is important. \\
  The third field writes the argument passed to the \texttt{set\_leds} routine,
  which is passed via \texttt{r0}, to the LEDs FU. The data value `3' would set
  both LEDs, for example. \\
  Notice that the registers are preceded by a `\textbackslash' character, this
  is optional and is simply a mnemonic to remind the programmer that the RF
  index bit-field is actually stored in the previous instruction, for RF reads.
  This fact would prevent any RF reads in the preceding instruction (\#1) from
  operating as expected.	\\
  The final field performs the same function as the first line, sets
  \texttt{com} to `1' for use with following instruction.
  \item This instruction acheives two purposes, the first field causes the
  program flow to branch back to the location after the instruction that called
  this routine (which means that the instruction within the branch delay slot
  after the instruction calling the \texttt{set\_leds} routine is executed
  twice). By convention, this location is stored within register \texttt{r15}. \\
  The last two fields of this instruction are used to restore the value of 
 the stack segment address register. This was modified at the start of this
 routine, and subsequent instructions executed after the completion of this
 routine will likely use this register and expect it to have the correct
 contents. This is another programming convention, as is using \texttt{r12} to
 store the stack segment register value.
  \item This is the branch delay slot instruction, but nothing is needed to be
  done in this slot so this is a simple \texttt{nop} instruction.
\end{enumerate}


\subsection{Functional Units}
Since TTA processors can typically execute multiple operations simultaneously,
the FUs need to be able to operate in parallel. TTA16 has bitwise logic unit, a
multiply unit, a subtractor, a branch unit, a Wishbone interface unit, and a
Machine Special Register (MSR) unit. Upto three\footnote{A write to
\textit{cmp} `register' causes both the bitwise and subtract FUs to be
triggered. This is so the zero flag (\textit{ZF}) and borrow flag (\textit{BF})
are modified simultaneously, which is useful for conditional branching.} of
these can be triggered, and therefore execute in parallel, each clock cycle.
This is because there are three data-transports that can write to trigger
registers.

Table~\ref{TTAPROG_Registers} shows the which data transports can
access which registers (or aliases), and Table~\ref{TTAPROG_FU_Regs} shows
which FUs each register (and its aliases) maps to. For example, the branch unit
can only be accessed from transport-0, and the \textit{COM} transport is also
connected to the operand register port of most of the FUs, as well as all the
other transports. In practice, this tends to be the most heavily used transport.

\begin{table}[h!]
\begin{center}
\begin{tabular}{l | l | l | l}
Name			& Inputs		& Outputs	& Aliases/Modes	\\
\hline
Multiply		& com, mul$_t$	& plo, phi	& mul	\\
Branch			& bra$_t$		& pc		& bra, jz, jnz, jb, jnb	\\
Subtract		& com, sub$_t$	& diff		& sub, sbb, cmp$^+$	\\
Bitwise Ops		& com, bits$_t$	& bits		& and, or, xor, cmp$^+$	\\
Wishbone		& mem, ad$_t$	& mem		& rad, wad	\\
MSR				& com, msr$_t$	& N/A		& msr	\\
\end{tabular}	\\
\footnotesize
$^+$See the ALU section and its associated footnote.
\normalsize
\end{center}
\caption[Functional unit summary]{Functional unit summary.}
\label{TTAPROG_FU_Regs}
\end{table}


\subsubsection{Arithmetic Logic Unit}
Traditional RISC processors have an ALU (Arithmetic Logic Unit) that perform
arithmetic and logical operations on integers, but TTAs use a multiple FUs
attached to data transports to achieve the same functionality. This is so
multiple integer operations can occur simultaneously without requiring multiple
ALUs.

The ALU functionality of TTA16 is obtained from just two FUs, the bitwise
operation FU, and the subtract FU. The missing logical operators from these two
FUs are the common left and right shift operators. Explicit shift FUs are not
provided but the same functionality can be obtained using the
multiplier\footnote{Multiplying by two is the same as a left shift. Multiplying
by 32,768 and discarding the low 16 bit word of the 32 bit product is the same
as a right shift of one place.}.


\subsubsection{Memory Segment Registers}
\label{TTAPROG_MSR}
% TODO: Actually Machine Special Registers
TTA16 has two parameterisable MSRs (Memory Segment Registers), each up to
16-bits in width, to allow memory addresses greater than 16-bits wide to be
supported. The memory segment value is concatenated with the memory pointer
value to give a total width of up to 32-bits, and with two bytes per address,
this means up to 8 GB is addressable with this architecture.

The two MSRs are called the Data Segment (DS) and the Stack Segment (SS)
registers which indicates their possible uses, though no such limit upon
such use is incorporated within the hardware. When performing large block
copies, for example, they both could be used as data registers.

By default, though this is set by the assembler/convention, not by the
hardware, the DS is used with load and store instructions. To use SS, instead
of DS, the suffix `.sf' is appended to the load/store instruction.


\section{OpenVGA Memory-Mapped I/O Peripherals}
OpenVGA memory space is divided into two regions, Memory-Mapped I/O (MMIO) or
SDRAM. Table~\ref{TTAPROG_MMIO} lists the MMIO devices implemented within
OpenVGA, and the associated segment to access the device.

\begin{table}[h!]
\begin{center}
\begin{tabular}{l | c}
I/O Device	&	Address Segment	\\
\hline
CRTC		&	0x0040	\\
SPROM		&	0x0044	\\
LEDS		&	0x0048	\\
DMA			&	0x0050	\\
Cache Flush	&	0x0058	\\
\end{tabular}
\end{center}
\caption[TTA16 memory-mapped I/O modules and segments]{TTA16 memory-mapped I/O
modules and there corresponding address segments.}
\label{TTAPROG_MMIO}
\end{table}


\subsection{Cache Flush Peripheral}
\label{TTAPROG_Cache_Flush}

\mmodule{Patrick Suggate}{wb\_cache\_flush}%
{Flushes all data from the data cache.}%
{/rtl/cache/wb\_cache\_flush.v}%
{/sim/cache/wb\_cache\_flush\_tb.v}{GPL}

It is possible for the contents of the cache and the contents of OpenVGA's main
memory to become incoherent. For example, if OpenVGA's processor is performing a data
conversion, reading from an arbitrary block `A', and writing to block `B', and a
PCI write transaction modifies the contents of block `A' in main memory, but data from
this location has already been cached, the contents of main memory and the data
cache are now incoherent.

The solution used within OpenVGA is for the processor to manually flush the data
cache, so that fresh data is retrieved from main memory. This peripheral allows
the manual flushing of each of the 16 cache-lines.

Writing to this device 16 times, to each of the 16 cache-line addresses,
completely flushes the cache. Any value for the write data will do since it is
not used by this peripheral. The 16 addresses which invalidate the cache-lines
are: \{0x0000, 0x0020, \ldots, 0x01e0\}


\subsection{CRT Controller}
\label{TTAPROG_CRTC}

\mmodule{Patrick Suggate}{wb\_crtc}%
{Generates the timing signals for a CRT or LCD display.}%
{/rtl/video/wb\_crtc.v /rtl/video/crtc.v}%
{/sim/video/crtc\_tb.v}{GPL}

% \begin{tabular}{l l}
% \textbf{Modules:}		& wb\_crtc	\\
% \textbf{Related Files:}	& /rtl/video/wb\_crtc.v, /rtl/video/crtc.v	\\
% \textbf{Testing Files:}	& /sim/video/crtc\_tb.v	\\
% \end{tabular}

The CRTC generates the display timing signals using the character clock, which
runs at one eighth of the dot-clock frequency. This means that the horizontal
timing signals (the first four within Table~\ref{TTAPROG_CRTC_Ports}) need to
be multiplied by eight to obtain the values in dot-clocks/pixels.

The default values correspond to the 640x480, with a refresh rate of 60Hz, video
mode, and the dot-clock is 25 MHz. These values are a running total so each value
in the table, for either the horizontal or vertical totals, is the previous value
plus the value of the current parameter. For example, the width in characters of
the horizontal back porch is six, or 48 pixels, and this is added to the 11
character clock ticks of the horizontal sync signal, to give the running total of
17 ticks. Other modes are listed in Table~\ref{VIDEO_Modes_Table}.

\begin{table}[h!]
\begin{center}
\begin{tabular}{c r l}
CRTC Port	& \multicolumn{1}{c}{Default}	&	Register Function	\\
			& \multicolumn{1}{c}{Value}	&						\\
\hline
0			&	11			&	Hsync duration		\\
1			&	17			&	Hroiz. back porch	\\
2			&	97			&	Horiz. active		\\
3			&	99			&	Hroiz. front porch	\\
4			&	1			&	Vsync duration		\\
5			&	34			&	Vert. back porch	\\
6			&	514			&	Vert. active		\\
7			&	524			&	Vert. front porch	\\
\end{tabular}
\caption[CRTC ports and their defaults values]{CRTC ports and their defaults
values.}
\label{TTAPROG_CRTC_Ports}
\end{center}
\end{table}


\subsection{DMA Unit}
\label{TTAPROG_DMA}

\mmodule{Patrick Suggate}{wb\_dma}%
{Writes buffered data to main memory in burst-mode.}%
{/rtl/wb\_dma.v /rtl/util/pre\_read.v}%
{/sim/wb\_dma\_tb.v /sim/xilinx/RAMB16\_S18\_S36.v}{GPL}

Since all direct memory writes performed by the CPU pass through the cache,
which is a write-through design, they have a high latency penalty. Since the
memory bus is shared by the PCI and display redraw units too, numerous CPU
memory writes will significantly increase memory congestion. This is because a
CPU to memory write is 16-bits wide and atomic, whereas the memory controller
is designed for 32-bit wide burst transfers.

To increase CPU write performance, and reduce memory bus contention, the CPU
can write data to a buffer, which later can have its contents written, as
32-bit wide bursts, to main memory. This buffer, a DMA, stores up to 2 kB of
data which can be wrtten, as a sequential burst, to any location in main memory.

To use the DMA, the destination segment and pointer registers are set, bit
masks in the control register allow byte enables to be set/cleared, and data is
written to the write data register. Once all desired data has been written to
the DMA, setting the start transfer bit in the control register initiates the
direct memory access.

The DMA does not modify the data in the buffer, and the buffer is implemented
as a circular queue, so the same 2 kB of data can be written multiple times,
like to clear the frame-buffer. Listing~\ref{TTAPROG_TTA16_memcpy} demonstrates
using the DMA to achieve functionality similar to the C programming
laguage library function \textit{memcpy}.

\begin{center}
\begin{tabular}{c c l}
DMA Port	& \multicolumn{1}{c}{Default}	&	Register Function	\\
			& \multicolumn{1}{c}{Value}	&						\\
\hline
0	& 0	&	Write data			\\
1	& 0	&	Address pointer		\\
2	& 0	&	Address segment		\\
3	& 0	&	Control register	\\
\end{tabular}
\end{center}


\section{Assembly Coding Conventions}

% \subsection{Register Use Guidelines}
All 16 of the TTA16, and RISC16, registers that are within the RF are general
purpose, but several programming conventions were used when writing routines.
These are guidelines concerning register usage and are listed in
Table~\ref{TTAPROG_Register_Conventions}. Registers zero through three are
for argument passing, and returning, and if more than four arguments are
required, they should be passed on the stack. These registers can be modified
without backing up first, but all other should be saved if modified. This is
not strictly enforced but obeying them will ensure routines work as expected.

Stack operations are somewhat clumsy\footnote{A future version of TTA16 would
probably have a stack FU implemented using a BRAM, and with MFSRs as the
incrementers and decrementers. This would be fast while using very few
general-purpose FPGA logic resources.}, requiring the stack segment to be set,
and explicit stack-pointer incrementing and decrementing. Passing arguments as
pointers to arrays would probably be more efficient if possible.

\begin{table}[h!]
\begin{center}
\begin{tabular}{c | c l}
Name& Free to	& Usage \\
	& \multicolumn{1}{c}{Use?}	&		\\
\hline
r0	& Y	& Parameter 0 for function calls \\
r1	& Y	& Parameter 1 for function calls \\
r2	& Y	& Parameter 1 for function calls \\
r3	& Y	& Parameter 3 for function calls \\
% \hline
r4-r11	& B	& Backup before modifying, restore afterwards \\
% \hline
r12	& N	& Stack Segment \\
r13	& N	& Zero Register \\
r14 & N & Stack Pointer \\
r15	& N	& Link Register	\\
\end{tabular}
\caption[Register usage conventions]{Register usage conventions.}
\label{TTAPROG_Register_Conventions}
\end{center}
\end{table}


\section{Executing the Generated Object Code}

The TTA16 assembler reads in an assembly file (`.s') and produces an ASCII text
output file (`.out'). The format of the output file contains many lines of
the form:
\begin{center}
$<address>:<instruction>$
\end{center}
Each field is ASCII encoded hexadecimal, without a prefix or suffix. An
additional processing step is required to get this object code to the processor
so it can be executed. There are three available options:
\begin{enumerate}
  \item Include the object code in the synthesis step so that it is placed into
  a BRAM.
  \item Append the object code to the ROM file so that it can be retrieved after
  FPGA initialisation, and transferred to main memory.
  \item The host PC can transfer a binary image to the main memory, via the PCI
  bus.
\end{enumerate}
The last two options require that the processor explicitly fetch instructions
from main memory and then store them in the local instruction BRAM. The code
can then be executed.

For the first option, placing code within the BRAMs at synthesis, a
post-processor tool \texttt{out2v.py} was written that converts a \texttt{`.out'}
file into a file that is included into a Verilog source file containing a BRAM.
This sets the contents of the BRAM at synthesis. This code is ready to be
executed as soon as the FPGA has finished initialising and clocks stabilised.
Because TTA16 has two BRAMs, one containing the upper 16-bits of each
instruction, the other the lower 16-bits. The post-processor can generate two
files to satisfy this. For the generated file \texttt{test.out} the following two
commands are issued:
\begin{center}
\tt
out2v.py -n 2 -s 0 test.out tta\_asm0.v	\\
out2v.py -n 2 -s 1 test.out tta\_asm1.v
\rm
\end{center}
The two generated files can then be included, using the \texttt{`include}
directive, in the initialisation field of a BRAM (see the source file
\texttt{/rtl/cpu/tta16/tta16.v} for an example).


\subsection{TTA16 Programming Example: `memcpy'}
The data cache used with TTA16 is a write-through design, and since the SDRAM
is shared with PCI and video redraw modules too, many write-through accesses
will reduce memory throughput. This is because each write is a 16-bit atomic
write, and the memory controller is most efficient with 32-bit burst reads and
writes. To achieve high performance, the TTA16 can (and should) write blocks of
data using the DMA controller.

The DMA controller is connected to the TTA16 via the I/O bus, not the memory
bus, so TTA16 can write data to the DMA controller without causing memory
contention with the other modules. Once all desired data has been transferred
to the DMA controller's local memory (2 kB), a write command can be issued
which will cause all stored data to be written to the SDRAM. The write is a
burst, and 32-bits wide, so the efficiency of this technique is good.

The following is a programming example showing how to program OpenVGA's DMA
controller to perform a memory copy. The data cache will perform data
prefetches using 32-bit bursts as well, so all memory accesses are 32-bit wide,
burst transfers.

The following is a `memcpy' routine, written in the C programming language, of
how a naive memory copy routine could be written:
\begin{figure}[h!]
\begin{lstlisting}[language=C]
void* memcpy(void* dst, void* src, int n)
{
	while(n--)
		(int*)*dst++	= (int*)*src++;
	return	dst;
}
\end{lstlisting}
\caption[C `memcpy' Example]{A memory copy routine that copies `n' integers
from `src' to `dst'.}
\label{TTAPROG_C_memcpy}
\end{figure}


Similar code written in TTA16 assembly language is shown in
Figure~\ref{TTAPROG_TTA16_memcpy}

\begin{figure}[h!]
\begin{center}
\footnotesize
\verbatiminput{source/tta16_memcpy.tex}
\normalsize
\caption[TTA16 Assembly `memcpy' Example]{A memory copy routine, written in
TTA16 assembly language, using the DMA module of OpenVGA.}
\label{TTAPROG_TTA16_memcpy}
\end{center}
\end{figure}

\chapter{RISC16 Assembly Language Programming Guide}
\label{RISCPROG}

RISC16 is a simple 16-bit RISC processor and this appendix is an introductory
programming guide for it. Contained in this guide are some code fragments and
simple routines, and an explanation of the RISC16 registers and instructions.
Related topics that are covered in significant detail in the TTA16 programming
guide (see Appendix~\ref{TTA_Programming}) will be covered only briefly here.


\section{Tools: The RISC16 Assembler}

\filedescript{Patrick Suggate}{r16asm.py}{A simple RISC16
assembler.}{/src/r16asm.py, /src/CodeCleaner.py, /src/AsmParse.py,
/src/Emit.py}{/src/fw\_risc16/*.S}{GPL}

% \begin{tabular}{l l}
% \textbf{Name:}			& r16asm.py	\\
% \textbf{Related Files:}	& /tools/r16asm.py, /tools/CodeCleaner.py,	\\
% 						& /tools/AsmParse.py, /tools/Emit.py	\\
% \textbf{Testing Files:}	& /src/fw\_risc16/*.S	\\
% \end{tabular}

An assembler, written in Python, is used to assemble RISC16 assembly files. Since
the RISC16 has just one execution thread, and some special instructions, the TTA
assembler was not ideal for RISC16 . The biggest problem was using the
\textit{i12} instruction with labels (for long jumps). This was solved with
\textit{r16asm} as it can evaluate Python mathematical expressions involving
bitwise operators, arithmetic operators, numbers, and named constants.

Example assembler usage:
\begin{verbatim}

        r16asm.py in_file.S -o out_file.out
\end{verbatim}
And then to make a \texttt{.v} file to include in synthesis:
\begin{verbatim}

        out2v.py -n 1 -s 0 out_file.out risc_asm.v
\end{verbatim}


\section{Instruction Format Overview}

Most RISC16 instructions are of a typical two-operand RISC form, though there are
several single-operand and three-operand instructions too. The supported RISC16
instruction formats are listed in Figure~\ref{RISCPROG_Instruction_Formats}.

\begin{figure}[h!]
\begin{center}
\includegraphics[width=12cm]{diagrams/risc16_instr.eps}
\caption[RISC16 instruction formats]{RISC16 instruction formats.}
\label{RISCPROG_Instruction_Formats}
\end{center}
\end{figure}

The following is an \texttt{rr}-format, (see
Figure~\ref{RISCPROG_Instruction_Formats}) subtract instruction where \texttt{r0}
gets the value of \texttt{r0} minus \texttt{r2}:

\begin{center}
\begin{minipage}{0.5\linewidth}
\texttt{sub r0, r2}
\end{minipage}
\end{center}

The destination register (in this case \texttt{r0}) precedes the source registers
and immediate constants in the operand list of RISC16 instructions. The
two-operand RISC16 instructions use the first operand as a source and
destination.

This is an example of a three operand (\texttt{rri}-format) instruction:

\begin{center}
\begin{minipage}{0.5\linewidth}
\texttt{subi r0, r0, 1}
\end{minipage}
\end{center}

With this format, all three operands are specified, even though \texttt{r0}
repeated in this case (it is typically different and is encoded in a separate
bit-field). This instruction subtract \texttt{1}, a 4-bit, signed immediate
constant, from \texttt{r0}. This is equivalent to a decrement of \texttt{r0}, and
it has short-hand version that is also accepted by the assembler:

\begin{center}
\begin{minipage}{0.5\linewidth}
\texttt{dec r0}
\end{minipage}
\end{center}

This instruction has exactly the same function and encoding as the preceding
\texttt{subi} example and this is just a convenient short-hand form. There is
also a corresponding \texttt{inc} mnemonic, which increments a register, as well.

Most ALU operations do not have a three operand form so to use immediate
constants, there is another instruction format, \texttt{ri}:

\begin{center}
\begin{minipage}{0.5\linewidth}
\texttt{sub r0, 1}
\end{minipage}
\end{center}

The result, stored in \texttt{r0}, is \texttt{1} minus \texttt{r0}. Note that
this is the reverse order to \texttt{subi}. This allowed a negate short-hand
instruction to be easily implemented too, simply subtract the desired register
from zero.

If immediate constants are required that are greater than can be stored as a
signed, 4-bit number (-8 to 7), an instruction prefix can be used:

\begin{center}
\begin{minipage}{0.5\linewidth}
\texttt{i12 0x123}	\\
\texttt{sub r0, 0x4}
\end{minipage}
\end{center}

This instruction sequence uses the 16-bit value 0x1234 (the `0x' is to indicate
that the following value is represented as hexadecimal) as the immediate
constant. This immediate prefix can be used with all \texttt{rri} instructions,
like \texttt{subi}, and \texttt{ri} instructions, like the \texttt{sub} example
above.

Conditional jump instructions are of the form:

\begin{center}
\begin{minipage}{0.5\linewidth}
\texttt{jxx	somewhere}
\end{minipage}
\end{center}

The \texttt{jxx} represents any one of the eight supported jump types
(see~\ref{RISCPROG_Jumps}). The \texttt{somewhere} is a destination address
(within the instruction SRAM), and it can be a number, a label, or a simple
mathematical expression (which can contain labels as these will be evaluated).

Load and store instructions are of type \texttt{rri} and have several syntaxes.
An example of the syntax used later in this appendix is:

\begin{center}
\begin{minipage}{0.5\linewidth}
\texttt{lw.d r0, [r1-3]}
\end{minipage}
\end{center}

The mnemonic \texttt{lw.d} means \texttt{l}oad-\texttt{w}ord, using the
\texttt{d} memory segment\footnote{There are two segment registers, \texttt{d}
and \texttt{s}. Their names were chosen to represent either \texttt{s}ource and
\texttt{d}estination segments, or \texttt{s}tack and \texttt{d}ata segment
registers, depending on the situation. This is purely a way to remember their
names, and make code more readable, as these registers function identically.}.
The square brackets following \texttt{r0} are simply to help the programmer
remember that the second and third operands are used to calculate a memory
address, from which a 16-bit word is loaded. The same instruction syntax of
\texttt{subi} can be used instead if desired, as these instructions share the
same encoding. The same load instruction could then be rewritten as:

\begin{center}
\begin{minipage}{0.5\linewidth}
\texttt{lw.d r0, r1, 3}
\end{minipage}
\end{center}

Note that the sign has changed, but this is exactly the same operation. This is
because the ALU performs a subtract to calculate the memory address for load and
store instructions. This has to be remembered when using the \texttt{i12}
instruction prefix as well.


\subsection{RISC16 Registers}

RISC16 has 16 general purpose registers. While they all are functionally
identical, by convention several of the registers have function. There is no
hardware restriction on which registers can be used, it is simply meant to
increase interoperability between assembly routines.

\begin{table}
\begin{center}
\begin{tabular}{c | c | l}
Name& Free to	& Usage \\
	& Modify?	&		\\
\hline
r0	& Y	& Parameter 0 for function calls \\
r1	& Y	& Parameter 1 for function calls \\
r2	& Y	& Parameter 1 for function calls \\
r3	& Y	& Parameter 3 for function calls \\
\hline
r4-r11	& B	& Backup before modifying, restore afterwards \\
\hline
r12	&	N	& Stack Segment \\
r13	&	N	& Zero Register \\
r14 &	N	& Stack Pointer \\
r15	&	N	& Link Register	\\
\end{tabular}
\end{center}
\end{table}


\subsection{ALU Instructions}

The general form of RISC16 arithmetic instructions is two input values, one
always a register, and the other a register or immediate, and a destination
register. For most arithmetic instructions (except \texttt{subi} and
\texttt{subic}) the destination register has to be one of the source registers as
well.

The list of available ALU instructions is shown in Table~\ref{RISCPROG_ALU}.
Many of these instructions have two forms, one which modifies the processor
condition-code flags, and one that does not. It is usually as simple as appending
a \texttt{c} to the end of an ALU instruction to cause it to modify the flags.
Bitwise operations only set the Zero Flag, and subtract-based operations
(\texttt{sub, sbb, inc, dec, subi,} and \texttt{neg}) only set the Negative Flag
(NF) and Borrow Flag (BF). The \texttt{cmp} instruction modifies all of these
three flags\footnote{The \texttt{cmp} instruction performs both an XOR and a
subtract on the operands.}.

Some example ALU instructions:
\begin{center}
\begin{minipage}{0.5\linewidth}
\begin{tabular}{l l}
\tt neg	& \tt r0	\\
\tt and	& \tt r1, -4	\\
\tt cmp	& \tt r2, 0x2	\\
\tt cmp	& \tt r3, r15	\\
\tt incc& \tt r4	\\
\tt subi& \tt r5, r6, 0	\\
\end{tabular}
\end{minipage}
\end{center}

\begin{table}[h!]
\begin{center}
\begin{tabular}{l l l l}
\multicolumn{1}{c}{Normal} & \multicolumn{1}{c}{Set Flags} &
	Encoding & Description \\
	\hline
	\tt subi & \tt subic&	\tt rri	& Subtract, 3-operand format	\\
	\tt inc & \tt incc	&	\tt rri	& Increment	\\
	\tt dec & \tt decc	&	\tt rri	& Decrement	\\
	\tt sub & \tt subc	&	\tt rr, ri	& Subtract	\\
	\tt sbb & \tt sbbc	&	\tt rr, ri	& Subtract using the borrow flag	\\
	\tt neg & \tt negc	&	\tt ri	& Negate	\\
	\tt N/A	& \tt cmp	&	\tt rr, ri	& Compare	\\
	\tt and & \tt andc	&	\tt rr, ri	& Bitwise logical AND	\\
	\tt nand & \tt nandc&	\tt rr, ri	& Bitwise logical NAND	\\
	\tt or & \tt orc	&	\tt rr, ri	& Bitwise logical OR	\\
	\tt xor & \tt xorc	&	\tt rr, ri	& Bitwise logical XOR	\\
	\tt mull & \tt N/A	&	\tt rr, ri	& Multiply, store low 16-bits	\\
	\tt mulh & \tt N/A	&	\tt rr, ri	& Multiply, store high 16-bits	\\
	
\end{tabular}
\end{center}
\caption[ALU instructions that can optionally set condition codes]{RISC16 ALU
instructions that can optionally set condition code flags.}
\label{RISCPROG_ALU}
\end{table}


\subsection{Processor Condition Code Flags}
RISC16 has three processor state flags which are set by the ALU. These are
borrow, negative, and zero flags (BF, NF, ZF). When an instruction has an
additional \texttt{c} appended to the instruction mnemonic the processor
condition codes are then to be modified by that instruction. The processor flags
are used with conditional jumps.


\subsection{Multiply Instructions}
Only 16-bit unsigned multiplication is directly supported, producing a 32-bit
product. Only either the upper or lower 16-bits of the 32-bit product can be
written back to the RF, not both. The \texttt{mull} instructions stores the lower
16-bits of the product in the destination register, the \texttt{mulh} stores the
upper 16-bits. The following assembly fragment is an example of the syntax:

\begin{center}
\begin{minipage}{0.5\linewidth}
\texttt{mull r0, 0x4}
\end{minipage}
\end{center}


\subsection{Program Flow Control}
\label{RISCPROG_Flow_Control}

RISC16 takes three cycles to perform a jump or branch, but unconditional branches
typically use the \texttt{i12} instruction, adding one more clock cycle. This is
to set the upper bits for the branch address. This approach does not give a
significant performance penalty since conditional branching is far more common
than unconditional jumps~\cite{mcfarland2003md}.

Conditional jumps have the 10-bit destination address encoded into the
instruction word, requiring just one instruction. Register indirect branches
takes just one instruction as well, since the entire destination address is
stored in a register. All of the supported jump and branch operations are listed
in Table~\ref{RISCPROG_Jumps}.

\begin{table}[h!]
\begin{center}
\begin{tabular}{l l l}
\multicolumn{1}{c}{Mnemonic} & Encoding & Description \\
	\hline
	\tt jnz	&	\tt bx	& Jump if ZF not set	\\
	\tt jz	&	\tt bx	& Jump if ZF set	\\
	\tt jl	&	\tt bx	& Jump if NF set	\\
	\tt jg	&	\tt bx	& Jump if NF not set	\\
	\tt jb	&	\tt bx	& Jump if BF set	\\
	\tt jbe	&	\tt bx	& Jump if BF and ZF set	\\
	\tt ja	&	\tt bx	& Jump if BF not set	\\
	\tt jae	&	\tt bx	& Jump if BF not set and ZF set	\\
	\tt br	&	\tt ri, rr	& Unconditional branch	\\
	\tt brl	&	\tt ri, rr	& Branch with link	\\
\end{tabular}
\end{center}
\caption[RISC16 branch and jump instructions]{RISC16 branch and jump
instructions.}
\label{RISCPROG_Jumps}
\end{table}


\section{Loads and Stores}

RISC16 has a minimalist FU that initiates transfers over the Wishbone bus, either
data loads or stores\footnote{Since the processor operates at much higher
frequencies than the system Wishbone bus, a data cache is used which also
functions as a synchroniser. It additionally changes the data width from 16-bits,
of RISC16, to 32-bits, of the system bus, and vice versa as well.}. Main memory
is accessed through a data cache connected between the processor and the system
Wishbone bus.

There are two instructions for initiating Wishbone bus operations, load word and
store word (`lw' and `sw' respectively). Each of these instructions causes the
processor to stall until an acknowledge is received from the Wishbone bus. This
potentially wastes cycles, but the trade-off is that data forwarding is not
needed for memory operations, so no dependency checking logic is needed either.

The ALU, specifically the subtract FU, is used to calculate the final address
from the supplied arguments. The format is always register minus an immediate,
but this can be zero (or the zero register which is \texttt{r13} by definition).
Since the ALU operation is subtraction, immediate operands have to be negated
when calculating the address. Listing~\ref{RISCPROG_Memcpy} contains examples of
these instructions.

The ALU bypass register is used for sending data to the wishbone bus. The only
valid source for this bypass register is \texttt{RS}, and there is no data
forwarding provided by hardware for this register either. Any potential data
hazards that are detected will result in pipeline interlocking.


\subsection{Memory Segment Registers}
RISC16 has two parameterisable, each up to 16-bits in width, MSRs (Memory Segment
Registers) to allow memory addresses greater than 16-bits wide to be supported.
The memory segment value is concatenated with the memory pointer value to give a
total width of upto 32-bits, and with two bytes per address, this means up to 8
GB is addressable with this architecture.

The two MSRs are called the Data Segment (DS) and the Stack Segment (SS)
registers which indicates their possible uses, though no such limit upon such use
is incorporated within the hardware. When performing large block copies, for
example, they both could be used as data registers.

By default, though this is set by the assembler/convention, not by the hardware,
the DS is used with load and store instructions. To use SS, instead of DS, the
suffix `.sf' is appended to the load/store instruction.


\subsection{OpenVGA MMIO Device Segments}
RISC16 and TTA16 are interchangeable when synthesising OpenVGA, and all
peripherals external to the processor are therefore identical. The MMIO devices
are the same for both RISC16 and TTA16 . For a complete list of MMIO devices, see
Appendix~\ref{TTA_Programming}, and the \textit{memcpy} listing later in this
section demonstrates how to access MMIO devices (see
Figure~\ref{RISCPROG_Memcpy}).


\section{Programming Example: `memcpy'}
\label{RISCPROG_Memcpy}
This code has the same functionality as the TTA16 assembly code in
Listing~\ref{TTAPROG_TTA16_memcpy}. The routine copies a block of memory from the
source to the destination. The DMA is used to improve memory throughput as it
allows the memory controller to operate in 32-bit wide, burst-mode reducing the
cycles lost due to multiple memory access-latency penalties.

\footnotesize
\verbatiminput{source/r16_memcpy.tex}
\normalsize
